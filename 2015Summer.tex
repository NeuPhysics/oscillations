\documentclass{tufte-handout}
%\geometry{showframe}% for debugging purposes -- displays the margins

\usepackage{amsmath}

% Set up the images/graphics package
\usepackage{graphicx}
\setkeys{Gin}{width=\linewidth,totalheight=\textheight,keepaspectratio}
\graphicspath{{graphics/}}

\title{Neutrino Oscillations in Vacuum and Matter \thanks{2015 Summer}}
\author[Lei Ma]{Lei Ma}
\date{May 15 2015}  % if the \date{} command is left out, the current date will be used

% The following package makes prettier tables.  We're all about the bling!
\usepackage{booktabs}

% The units package provides nice, non-stacked fractions and better spacing
% for units.
\usepackage{units}

% The fancyvrb package lets us customize the formatting of verbatim
% environments.  We use a slightly smaller font.
\usepackage{fancyvrb}
\fvset{fontsize=\normalsize}

% Small sections of multiple columns
\usepackage{multicol}

% code highlighting
\usepackage{listings}

\usepackage{minted}
\usepackage[utf8]{inputenc}
\usepackage[english]{babel}

% Provides paragraphs of dummy text

% These commands are used to pretty-print LaTeX commands
\newcommand{\doccmd}[1]{\texttt{\textbackslash#1}}% command name -- adds backslash automatically
\newcommand{\docopt}[1]{\ensuremath{\langle}\textrm{\textit{#1}}\ensuremath{\rangle}}% optional command argument
\newcommand{\docarg}[1]{\textrm{\textit{#1}}}% (required) command argument
\newenvironment{docspec}{\begin{quote}\noindent}{\end{quote}}% command specification environment
\newcommand{\docenv}[1]{\textsf{#1}}% environment name
\newcommand{\docpkg}[1]{\texttt{#1}}% package name
\newcommand{\doccls}[1]{\texttt{#1}}% document class name
\newcommand{\docclsopt}[1]{\texttt{#1}}% document class option name



% For quantum braket notation
\newcommand{\bra}[1]{\left\langle #1\right|}
\newcommand{\ket}[1]{\left| #1\right\rangle}
\newcommand{\braket}[2]{\langle #1 \mid #2 \rangle}
\newcommand{\avg}[1]{\left< #1 \right>}

\begin{document}

\maketitle% this prints the handout title, author, and date

\begin{abstract}
\noindent Notes for neutrino oscillations in vacuum and dense matter.
\end{abstract}



\section{Vacuum Oscillations}

Schrodinger equation is

\begin{equation}
i\partial_t \ket{\Psi} = \mathbf H \ket{\Psi},
\end{equation}

where for relativistic neutrinos, the energy is

\begin{align*}
\mathbf H &= \sqrt{p^2 + m^2} \\
&= p\sqrt{1 + \frac{m^2}{p^2}} \\
&\approx p(1 + \frac{1}{2} \frac{m^2}{p^2}).
\end{align*}


In general the flavor eigenstates are the mixing of the mass eigenstates with a unitary matrix $\mathbf U$, that is

\begin{equation}
\ket{\nu_{\alpha}} = \mathbf U_{\alpha i} \ket{\nu_i},
\end{equation}

where the $\alpha$s are indices for flavor states while the $i$s are indices for mass eigenstates.

To find out the equation of motion for flavor states, plugin in the initary tranformation,

\begin{equation}
i \mathbf U_{\alpha i} \partial_t \ket{\nu_i} = \mathbf U_{\alpha i} \mathbf H^m_{ij} \ket{\nu_j}.
\end{equation}

I use index ${}^m$ for representation of Hamiltonian in mass eigenstates. Applying the unitary condition of the transformation,

\begin{equation}
\mathbf I = \mathbf {U^\dagger} \mathbf U,
\end{equation}

I get

\begin{equation}
i\mathbf U_{\alpha i} \partial_t \ket{\nu_i} = \mathbf U_{\alpha i} \mathbf H^m_{i j} \mathbf {U^\dagger_{j\beta}} \mathbf U_{\beta k} \ket{\nu_k},
\end{equation}

which is simplified to

\begin{equation}
i \partial_t \ket{\nu_\alpha} = \mathbf H^f_{\alpha \beta} \ket{\nu_{\beta}},
\end{equation}

since the transformation is time independent.




\subsection{2 Flavor States}


For 2 flavor neutrinos the Hamiltonian in the representation of propagation states,

\begin{equation*}
\mathbf H  = \begin{pmatrix}
E_1 & 0 \\
0 & E_2
\end{pmatrix} 
 = \begin{pmatrix}
p_1 + \frac{1}{2}\frac{m_1^2}{p_1} & 0 \\
0 & p_2 + \frac{1}{2}\frac{m_1^2}{p_2}
\end{pmatrix}.
\end{equation*}

The equation of motion in matrix form is

\begin{equation}
i\partial_t \begin{pmatrix}
\nu_1 \\ \nu_2 \end{pmatrix} = \begin{pmatrix}
p_1 + \frac{1}{2}\frac{m_1^2}{p_1} & 0 \\
0 & p_2 + \frac{1}{2}\frac{m_1^2}{p_2}
\end{pmatrix} \begin{pmatrix}
\nu_1 \\ \nu_2 \end{pmatrix} 
\end{equation}

The flavor eigenstate is a mixing of the propagation eigenstates,

\begin{equation}
\begin{pmatrix}
\nu_a \\ \nu_b \end{pmatrix} = 
\begin{pmatrix} \cos\theta_v & \sin\theta_v \\ -\sin\theta_v  & \cos\theta_v
\end{pmatrix} \begin{pmatrix}  \nu_a \\ \nu_b
\end{pmatrix}
\end{equation}


Denote the rotation matrix using $\mathbf U$, the transofmation can be written as

\begin{equation}
\ket{\nu_{\alpha}} = \mathbf U_{\alpha i} \ket{i},
\end{equation}

where $\alpha$ is for the flavor eigenstates and $i$ is for the mass eigenstates.





\subsection{3 Flavor States}





\section{Oscillations in Dense Medium}






\end{document}
